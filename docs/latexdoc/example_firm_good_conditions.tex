\documentclass[a4paper]{article}
\usepackage{fancyvrb}
\title{Example: Firm-Bank relationship in good conditions}
\begin{document}
\maketitle



\subsubsection{Good conditions}
	\subsubsection*{Step 1: interests and repayments}
	Suppose we start from the same conditions as in the bad cases.

	There are no differences in the evolution of this step, so that the situation at the end of this step is the same:
\begin{verbatim}
bank 1 account =   10  demanded credit =    0 allowed credit =    0
bank 2 account = -150  demanded credit = -150 allowed credit = -130 
bank 3 account =  -50  demanded credit =  -50 allowed credit =  -50
\end{verbatim}


	\subsubsection*{Step 2: refunding}
	Good conditions here means that the firm realized a profit and thus has a positive cash on hand, say

	\verb+cashOnHand = 50+

20 of them are used to refund the bank. So, now, the situation is 

\begin{verbatim}
bank 1: account =   10  demanded credit =    0 allowed credit =    0 unpaid =  0
bank 2: account = -130  demanded credit = -150 allowed credit = -130 unpaid =  0
bank 3: account =  -50  demanded credit =  -50 allowed credit =  -50 unpaid =  0

cashOnHand = 30
\end{verbatim}



	\subsubsection*{Step 3: account resetting}

\begin{verbatim}
bank 1: account =   10  demanded credit =    0 allowed credit =    0 unpaid =  0
bank 2: account = -130  demanded credit =    0 allowed credit =    0 unpaid =  0
bank 3: account =  -50  demanded credit =    0 allowed credit =    0 unpaid =  0

cashOnHand = 30
\end{verbatim}

	\subsubsection*{Step 4: set desired credit}
	Imagine now, the firm had a production capital equal to 100 before starting production. 
	
	Suppose production depreciates capital to 95.

	Furthermore, suppose the firm expects an increase of demand and wants to increase its production capital to 110.

	So, to bring production capital to the desired level 15 is needed.

	Checking financial resources available internally the firm conclude that it can achieve the objective without asking to banks. 

	Furthermore there is an inconsistency in firms bank accounts: the positive one should be used to reduce the negative ones. The software at this stage withdraws positive bank accounts and store them in the variable \verb+financialResourcesInBankAccounts+:

\begin{verbatim}
bank 1: account =    0  demanded credit =    0 allowed credit =    0 unpaid =  0
bank 2: account = -130  demanded credit =    0 allowed credit =    0 unpaid =  0
bank 3: account =  -50  demanded credit =    0 allowed credit =    0 unpaid =  0

cashOnHand = 30 financialResourcesInBankAccounts = 10
\end{verbatim}

	\subsubsection*{Step 5: credit supply}
In this step no change is performed because credit was not asked.

	\subsubsection*{Step 6: adjust production capital and banks accounts}

	Production capital is adjusted by using internal financial resources. 

	So, after this step we have

	\verb+productionCapital=110+

	\verb/cashOnHand + financialResourcesInBankAccounts = 25/

	finally, the residual internal refunds are used to improve the ``worst'' bank account:

\begin{verbatim}
bank 1: account =    0  demanded credit =    0 allowed credit =    0 unpaid =  0
bank 2: account = -105  demanded credit =    0 allowed credit =    0 unpaid =  0
bank 3: account =  -50  demanded credit =    0 allowed credit =    0 unpaid =  0

cashOnHand = 0 financialResourcesInBankAccounts = 0
\end{verbatim}

\end{document}




